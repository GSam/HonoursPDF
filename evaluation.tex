\chapter{Evaluation}\label{C:eval}
This chapter examines the correctness and practical use of the tool. The correctness of the code is measured predominantly with the testing currently supplied and without a large number of real world testers utilizing this code, it is hard to gauge the correctness of the code in general. Also, without general users of the tool, the analysis of practicality lacks depth, but again, the refactoring tool has been designed as a proof-of-concept to begin with.

[The types of considerations which could have been done are as follows:

Design considerations - well, workflow problems
Performance - how well does it perform as different aspects change:

1. Size of codebase

2. Number of refactoring locations - particularly for renaming + inlining

3. Type of renaming

Comparing to Eclipse or IntelliJ perhaps... but the language differences make it quite difficult, time complexity and what checking they actually perform isn't entirely clear. The functionality provided by my tool is also Rust specific in some cases. Inline vs in Java - mutability constraints.

User study - Why this is inappropriate? Could bring in users of Rust, but the language is not that popular still with only (XXX users). This wasn't a good idea, because finding the adequate amount of experience and users [target demographic]. There would also be nothing to compare to, besides perhaps manual refactoring... Availability + bias of people who know the language and its workings well.

Correctness testing - Formal, informal. Integration testing vs unit-testing. The compiler is tightly coupled with the refactoring tool and compare this to the modular refactoring definitions definable by aspects [obliviousness and ... something, refer to paper] or the Scala refactoring tool... The compiler is still quite unstable, although it has finally reached release status... Even amongst the six months or so working on this tool, so many API have changed under my feet. While unit testing would provide more assurance that the code written is correct, a major part of the underlying code is actually in the compiler and the refactoring tool has to trust that the code is already correct. Massive complexities, that just can't be nailed down. To continue to write unit tests in the compiler was an option, but one that clearly provided less benefit than testing the actual cases properly.


Describing the nature of the testing - What they actually do.. what failures there are still - to be listed in the appendix perhaps?

[Need to look more back at Scala thesis etc...]

% HYGIENE DEFINITION?
\section{Correctness of renamings}
Included in the code repository there is a number of tests to ensure that the refactoring tool functions as expected. Each test consists of a csv dump file, a Rust source file and if the refactoring is designed to be successful, an output Rust source file. The testing suite attempts to test both the cases where a refactoring should occur, and cases where it should not due to the variations in conflict types as outlined in Section \ref{S:different}. The tests use the inbuilt Rust functionality for providing testing and currently there around 20 different tests which should be doubled or tripled before the end of the project. Given the project time constraints, test cases appear to be the most effective way to ensure correctness. After reviewing the literature and noting the limited attempts at correctness, whether or not informal justifcations of correctness are satisfactory remains a big question.

\subsection{Local and global variables}
At the moment there exists several tests for const, static and normal local variables, both in successful and non-successful situations. In terms of the ability for this class of renamings to fail unexpectedly, the chances appear slight since variables lack the most dynanism and complexity (no dynamic dispatch for instance), particularly for local variables. As long as name resolution works correctly and the compilation process, there is little reason to doubt these renamings. During the process of testing, it was found that `static mut' global variables did not record their spans correctly due to an error in the save-analysis code. In particular, the span was recorded for the `mut' identifier as opposed to the actual name of the variable. Fixing this required a minor patch to the Rust compiler and this patch was upstreamed prior to the release of Rust 1.0.

%While the ownership system in Rust prevents aliases for variables, a variable can be redeclared in the same scope or in a child scope. 

\subsection{Methods and functions}
At the moment there are several tests for renaming methods defined with a trait and/or overridden by a inheriting struct. Tests address both cases of static dispatch and dynamic dispatch, however the full combination of outputs for methods and functions is unknown due to the limitations in the csv file description and the reliance on the use of declname as opposed to a proper id for dynamically dispatched methods. One known failure mode of the refactoring tool is when a function (or trait) declared within a scope. Prevented renamings include those on dynamically dispatched methods, handled by the compiler runs on each usage.

\subsection{Concrete types - structs and enums}
At the moment there are tests for both renaming of structs and enums with detection of namespace collisions (which are not in local scopes). The checking of namespace collisions also extend to the usage of `use' statements which allow a specific namespace to be added to the default and no need to additionally qualify some names. The renaming of concrete types does not extend to type aliases, or rather the extent to which the renamings function is still relatively unknown.

\subsection{Shared refactoring limitations}
All of the current refactorings rely on the fact that the save-analysis output is correct. During the process of this project, this has certainly been found to be false, although not to an extreme extent. Minor errors in processing have meant that little known edge cases have occurred and would not have been found without the testing that has currently been undergone. In terms of testing of the save-analysis functionality, it is relatively scarce and would be difficult to implement considering the different possible combinations of expressions and items. Beyond errors in the save-analysis translation, errors in the compiler would also affect the ability to function correctly, but the compiler in general appears to feature more testing and the dependencies required by save-analysis are quite critical. On the other hand, lesser used and documented API like name resolution could be affected and this is a problem especially from the context of a third party tool and the first of its kind to use various APIs. This makes adoption of this code important by those who use Rust, and want to see a tool flourish.

With the save-analysis, there is also the limitation of the missing macros in the output. As a pending issue in the Rust compiler, the necessary plumbing required for macros could be implemented given enough resources and should eventually function together with save-analysis. Unfortunately, until then, any definitions of correctness must exclude the use of macros. 

\section{Correctness of inline-local}
definitely still a proof of concept, uses pretty printer to modify code... expands macros by accident (better approach may be to write a reverse source matching method). comments... although not location preserving...

\section{Correctness of inline-local}
pretty printer... with elide. trouble with using the pretty printer, but the function impl should not have any major issues (elimination of comments).


\section{Extent of refactorings}
%Given only refactorings concerning renaming have been tackled, the scope of the investigation is still severely limited. 

The refactorings tackled predominantly concern renaming, however, the addition of local variable inlining and lifetime elision and reification has allowed for a greater view of refactorings in Rust. In terms of the categories of refactorings mentioned by Fowler et al., only the very surface of method composition and simplifcation have been covered, but this projects underlying nature has meant notable differences and distinctness?. 

[On the other hand, we have renaming which forms one of the primary atomic units within a refactoring and allows composition.]

% Rework this section -- Referring back to Section \ref{S:refactorback}, at least one of code movement or extraction, data organization or operations within the trait hierarchy (compared with the classic inheritance hierarchy) should be investigated in order to provide a more comprehensive look at refactoring within the context of Rust. A majority of the issues seen so far have had more to do with the limitations of the compiler itself and how it was implemented than those purely of the language. If possible, a refactoring which interacts with the ownership system should provide an more interesting glimpse at the difficulties associated with Rust specifically.

\section{Examining the steps necessary to perform a refactoring}

Looking at what is necessary to invoke the refactoring tool we can determine a general description of steps required:

\begin{enumerate}
\item Compile a program with -Z save-analysis to produce a csv analysis file.
\item Either inspect the csv manually or run the refactoring library to determine the node id of the element you wish you alter.
\item Run the refactoring tool, choosing your desired refactoring. With a rename, the new name must be supplied, the node id and (when multi-file refactorings are implemented) the file.
\item Wait for the refactoring to occur and the compiler to run all the proper checks. 
\item Process the result and save the result to disk, if desired.
\end{enumerate}

Ideally, the flow of behaviour a user would want would be to simply identify the row and column and the refactoring to happen automatically. Not involving node id at all would be good, but this is necessary to adequately treat the tool as a library as opposed to a full fledged tool. Being able to easily identify row and column still requires some form of GUI tool and integration with such a tool would be desirable to reduce the amount of different parts required to perform a refactoring. 

Having to regnerate a csv file every time a refactoring has been made is definitely not desirable, although using the same csv file could only ever allow renamings to new names of the same length and renaming would have to occur on separate variables. Even though the compiler would probably generate the same AST with the same node ids (being deterministic) having different lengths in a new name would cause all the indexes into the code maps to be at the wrong offsets. Implementing some form of analysis cache could reduce user friction and reduce the amount of time spent compiling, but unless it managed all the changes in offsets correctly, as soon as a refactoring which significantly changed the output began, any existing analysis would still be out of date. Likely the best solution would be to correctly implement incremental compilation within the Rust compiler, which would allow less significant compile times and faster regeneration of analysis -- functioning basically like a cache. The lack of such behaviour is a shortcoming of the compiler and does appear to require significant structual changes but the benefits would extend further than just enabling better refactoring.

% This could be done outside of the refactoring library, in a persisting background process but on the other hand,

\section{Analyzing time taken or performance evaluation}
With only single file tests, the amount of time spent performing each refactoring is quite negligible. Compilation times for single files, especially of fairly trivial programs do not provide sufficient evidence of practical timings for performing a refactoring. This is definitely something which should be attempted before the project is completed.

Typically, the main consumption of compile time is with the analysis phase and generation of LLVM code. Within the refactoring tool, in all cases, the generation of LLVM code is avoided since only source code modifications are being performed and the analysis phase provides more than enough information for validity of a refactoring. While a rename refactoring requires checking every usage of a declared item using the compiler, in the `happy path' every usage should fail the compiler check during the early stages of parsing or name resolution. Only in more unlikely or unfortunate cases will a refactoring require any additional processing in analysis. In this regard, the compiler approach likely is not as bad as first theorized. On the other hand, the number of runs is still proportional to the number of usages but this is a direct consequence of the current limitations of the name resolution API.

\subsection{Time taken for a refactoring within a moderate-sized crate}
This should be a reasonable test for some metric of the effectiveness of the tool...

Things to compare against, the amount of usages for renamings, comparing the different renamings for speed differences, the time taken to reach analysis vs overall compilation times. 
I have some examples, but I think Nick should know best what crates might be decent to test with.

Offsets to negative performance:
The appearance of modular systems, Rust encouraging use of Crates.io and package management. Object-orientation as opposed to other paradigms? On the other hand, more modular systems mean more API and shared functions, types etc. This doesn't necessarily make a refactoring easier because once an API is exposed, it is usually not liable to change (cite Josh Bloch perhaps). Cross code-base changes are problematic, not just for Rust, but for programmers in general.

