%% $RCSfile: proj_report_outline.tex,v $
%% $Revision: 1.2 $
%% $Date: 2010/04/23 02:40:16 $
%% $Author: kevin $

\documentclass[11pt
              , a4paper
              , twoside
              , openright
              ]{report}


\usepackage{float} % lets you have non-floating floats

\usepackage{url} % for typesetting urls

%
%  We don't want figures to float so we define
%
\newfloat{fig}{thp}{lof}[chapter]
\floatname{fig}{Figure}

%% These are standard LaTeX definitions for the document
%%                            
\title{Refactoring Rust Programs}
\author{Garming Sam}

%% This file can be used for creating a wide range of reports
%%  across various Schools
%%
%% Set up some things, mostly for the front page, for your specific document
%
% Current options are:
% [ecs|msor]              Which school you are in.
%
% [bschonscomp|mcompsci]  Which degree you are doing
%                          You can also specify any other degree by name
%                          (see below)
% [font|image]            Use a font or an image for the VUW logo
%                          The font option will only work on ECS systems
%
\usepackage[font,ecs,mcompsci]{vuwproject}

% You should specifiy your supervisor here with
\supervisor{Alex Potanin}
% use \supervisors if there is more than one supervisor

% Unless you've used the bschonscomp or mcompsci
%  options above use
\otherdegree{Bachelor of Engineering Honours}
% here to specify degree

% Comment this out if you want the date printed.
\date{}

\begin{document}

% Make the page numbering roman, until after the contents, etc.
\frontmatter

%%%%%%%%%%%%%%%%%%%%%%%%%%%%%%%%%%%%%%%%%%%%%%%%%%%%%%%

%%%%%%%%%%%%%%%%%%%%%%%%%%%%%%%%%%%%%%%%%%%%%%%%%%%%%%%

\begin{abstract}

This purpose of this document is to outline a new automated refactoring tool for the Rust programming language. The creation of the tool serves as an explorative process to investigate the current ability for Rust to provide automated refactorings and to draw comparisons to  efforts in providing refactoring tools for other languages. 

Beginning with a review of refactoring in general and Rust, this document describes an overview of the tool along with a selection of the major design decisions. Elaborating further, technical details and Rust-specifics are explored with respect to the capability and limitations of the tool. [Evaluating the tool, the correctness of each refactoring is examined, as well as the practical use of the tool.]

Regarding Rust specific aspects of refactoring, a number of further extensions to this tool are proposed, along with extensions to the Rust compiler which would help facilitate this work. In particular, [based on evaluation(?)], the issue of effiency of the tool should be further addressed and streamlining the process for convenient refactoring will contribute to the practical use of the tool.

% A short description of the project goes here.
% Rust nearing 1.0 release
% Modern memory-safe systems programming language
% But, tooling is still immature

% The task is to create a Rust refactoring tool that people could actually use. In doing so, this will help identify which refactorings are easy, which are hard and how this compares to other languages.
% Written in Rust
% Start off as a standalone library, but a more comprehensive tool may require more feedback from the compiler.  Development will be incremental and new features will be chosen based on current progress and expected difficulty. Time permitting, a GUI tool? 
% “Time” – Lots of concepts and code to understand
% No prior experience with Rust
% Unknown complexity, unknown support
% BUGS + documentation
% Up-streaming code – coding style, testing

\end{abstract}

%%%%%%%%%%%%%%%%%%%%%%%%%%%%%%%%%%%%%%%%%%%%%%%%%%%%%%%

\maketitle

\chapter*{Acknowledgments}\label{C:ack} 
% Any acknowledgments should go 
% in here, between the title page and the table of contents.  The 
% acknowledgments do not form a proper chapter, and so don't get a 
% number or appear in the table of contents.

This work is the culmination of effort and assistance made by a number of people. First of all, I would like to take the chance to thank my family.  To my parents for providing the support and guidance I always needed. To my sister for igniting the flame for pursuing Engineering and getting a chance to study a subject area I was genuinely passionate about.

For proposing this work and spending the time to mentor me throughout this year, I give thanks to Nicholas Cameron, Mozilla and the Rust community. 

My supervisor, Dr. Alex Potanin for providing support throughout this project and for reviewing the drafts of this report.

To Rust, and its suporters for bringing about the motivation for which this work has developed. Without the existing infrastructure brought forth by the community, this project would not have been possible.

Lastly to my friends and peers studying Engineering and Computer Science, without them I would have lost my way a long time ago.

\tableofcontents

% we want a list of the figures we defined
\listof{fig}{Figures}

%%%%%%%%%%%%%%%%%%%%%%%%%%%%%%%%%%%%%%%%%%%%%%%%%%%%%%%

\mainmatter

%%%%%%%%%%%%%%%%%%%%%%%%%%%%%%%%%%%%%%%%%%%%%%%%%%%%%%%

% individual chapters included here
\chapter{Introduction}\label{C:intro}
This work is concerned with the analysis and implementation of tools to support the automated refactoring of Rust programs. The resulting refactoring tool built is written in Rust and [available from Github under an MIT license]. The purpose of this chapter is to describe the overall work presented in this document and to begin to describe aspects of the motivation for pursuing this work. In Section \ref{S:rustintro} and Section \ref{S:refactorintro} the context of the project is introduced with Rust and refactoring. In Section \ref{S:contrib}, the contributions made by this work are described. In Section \ref{S:implemented}, the definite list of refactorings supported by the tool is described, along with a brief description of each. Lastly, in Section \ref{S:outline}, a summary of the subsequent chapters and the remaining document is given.

%In Chapter \ref{C:us} we explain how to use this document, and the \texttt{vuwproject} style. In Chapter \ref{C:ex} we say some things about \LaTeX, and in Chapter \ref{C:con} we give our conclusions.

\section{Introducing the Rust programming language}\label{S:rustintro}

% http://blog.rust-lang.org/2015/05/15/Rust-1.0.html

The Rust programming language, as of May 2015, reached the 1.0 milestone.  As a modern memory-safe systems programming language, the aim for the language is to provide reliable and efficient systems by combining the performance of low level control, with the convenience and guarantees of higher level constructs. All of this is achieved without a garbage collector or runtime and allows interoperability with no overhead with C. Rust enforces an ownership system to restrict the duplication of references through borrowing and lifetimes, while still aiming for `zero cost abstractions'. Using this techniques, Rust prevents dangling pointers and whole class of related issues concerning iterator invalidation, concurrency and more [ref https://doc.rust-lang.org/book/README.html].

Rust has been a number of years in the making [ref date?], but only with this public 1.0 release has many of the standard library API stabilized. In order to further present itself as a stable and mature language, there is still another component which has yet to be adequately tackled: tooling. Developers commonly use a rich set of tools to simplify the task of building complex software with IDE, editors and debuggers. Such tools can improve productivity and providing tools specific to a language may help to encourage adoption.

\section{The issue of refactoring}\label{S:refactorintro}
[Should this summarize the next chapter or not?] Refactoring is the act of performing functionality preserving code transformation. Traditionally, these transformations needed to be performed manually, but in recent years, a number of tools to aid and automate refactorings have arisen in many programming languages such as Java or C++ [cite some source?]. Manual transformations, including editor search-and-replace are potentially prone to error and so performing tool-assisted refactoring which guarantees some measure of correctness often provides much greater confidence in changes.

\section{Contributions of this work}\label{S:contrib}
The main purpose of this work is to produce a proof-of-concept refactoring tool which utilizes existing infrastructure made available by the Rust compiler. With continuing expectations for change within the language and compiler and a number of API are still undergoing stabilization, a refactoring tool would prove beneficial for users at risk of changes breaking their code underneath them. In particular, API changes which convert global functions e.g. foo(A, B) to A.foo(B), to methods acting on an object are common and providing hopefully this work provides some insight into the necessary work to achieve such a refactoring. 

Exploring the necessary barriers to implementing a refactoring tool, proposals or implementation of extensions to the Rust compiler are brought forth and help to identify current shortcomings with providing refactoring tools for the Rust programming language and work necessary to improve the capability of Rust to support refactoring. Additionally, the current extent of the ability for the compiler to facilitate refactoring is evaluated and a number of internal or unstable API are assessed. The compile driver API which provides the ability for Rust code to invoke the compiler is a primary example of code which has seen little real world usage. The name resolution module within the compiler is another key example used in this work for performing refactoring, being used in a manner not originally intended or imagined despite providing part of the functionality required for implementing such a tool.

This work presents an actual refactoring tool which may be run to provide a set of refactorings which may continue to be extended in the future. The list of refactorings supported are listed in the next section and described in detail in Chapter \ref{C:wd}. This work also attempts to provide initial evaluations as to the effectiveness and practicality of the current tool. Furthermore it aims to identify the most useful refactorings not currently supported by the tool and the most useful future changes required within the compiler for further iterations on the current work.

\section{Implemented refactorings}\label{S:implemented}
[Insert brief summary here]

\begin{itemize}
\item \bf{Renaming local variables} --
\item \bf{Renaming global variables} --
\item \bf{Renaming functions} --
\item \bf{Renaming methods} --
\item \bf{Renaming structs} -- 
\item \bf{Renaming enumerations} --
\end{itemize}

\section{Document outline}\label{S:outline}
[Should this be bullet pointed - should it be more descriptive?]
In Chapter \ref{C:wd} we explain the overall Rust refactoring tool, the major design decisions made [while in Chapter ?]/[and] we discuss the specifics for the implementation with a more indepth focus on Rust behaviour. In Chapter \ref{C:eval}, we discuss opportunities for evaluation of the tool and the current extent of testing for refactorings provided by the tool and overall correctness. In Chapter \ref{C:future}, we propose a number of further extensions to the tool, to the Rust compiler and in general, outline areas of future work while providing useful insights gained during the course of this work. Lastly, Chapter \ref{C:con} presents a summary, and gives conclusions about this work.
%\include{using}
\chapter{Background}\label{C:back} 
This chapter covers related background material which constitutes 
\chapter{Work Done}\label{C:wd}

\section{Design of Implementation}

% Throwing exceptions... no null types
% Error propogation system in terms of nesting, expectations on thread serialization

% Redeclarations made under the same scope
% Every approach seems incredibly intrusive.

% With API changing, one of the first to implement anything which interfaces in this particular manner.

[Split into two chapters?]

Because of macros... you want to raise conflicts with new declaration.
Why was it in Rust vs other languages? 

\subsection{}
[Taken from http://scala-refactoring.org/wp-content/uploads/scala-refactoring.pdf, to transcribe]
\begin{enumerate}
\item provide a user interface so that a specific refactoring can be discovered and
invoked from the IDE.
\item analyze the program under refactoring to find out whether the refactoring is
applicable  and  further  to  determine  the  parameters  and  constraints  for  the
refactoring.
\item transform the program tree from its original form into a new – refactored – form
according to the refactoring’s configuration.
\item turn this new form back into source code,  keeping as much of the original
formatting in place as possible and to generate code for new parts of the program.
\item present the result of the refactoring to the user – typically in the form of a patch –
and apply it to the source code
\end{enumerate}

[Figure describe work flow of refactoring or structure of the library]

caching to provide multiple refactorings in a single run and to only run the save-analysis where necessary again. 

Library 
\subsection{}

\subsection{}

\section{Specifics of the Implementation}
\subsection{}
\subsection{Unsuccessful attempts at refactorings?}
\subsubsection{Lessons learnt}
\chapter{Evaluation}\label{C:eval}
This chapter examines the correctness and practical use of the tool. The correctness of the code is measured predominantly with the testing currently supplied and without a large number of real world testers utilizing this code, it is hard to gauge the correctness of the code in general. Also, without general users of the tool, the analysis of practicality lacks depth, but again, the refactoring tool is designed to begin as a proof-of-concept.

% HYGIENE DEFINITION?

\section{Correctness of renamings}
Included in the code repository of there is a number of tests to ensure that the refactoring tool functions as expected. Each test consists of a csv dump file, a Rust source file and if the refactoring is designed to be successful, an output Rust source file. The testing suite attempts to test both cases where a refactoring should occur, and cases where it should not due to the variations in conflict types as outlined in Section \ref{S:}. The tests use the inbuilt Rust functionality for providing testing and currently there are [19] different tests.

\subsection{Local and global variables}
[At the moment several tests for const, static and normal local variables, both in successful and non-successful situations. In terms of the ability for these to fail unexpectedly, the chances appear slight since variables lack the most dynanism and complexity (no dynamic dispatch for instance), particularly for local variables. While there are no aliases for variables, a variable can be redeclared in the same scope or in a child scope. But as long as name resolution works correctly and the compilation process, there is little reason to doubt these renamings. During the process of testing, it was found that `static mut' global variables did not record their spans correctly due to an error in the save-analysis code. In particular, the span was recorded for the `mut' identifier as opposed to the actual name of the variable. Fixing this required a minor patch to the Rust compiler and this patch was upstreamed prior to the release of Rust 1.0.]

\subsection{Methods and functions}
[At the moment there are several tests for renaming methods defined with a trait and/or overridden by a inheriting struct. Tests address both cases of static dispatch and dynamic dispatch, however the full combination of outputs for methods and functions is unknown due to the limitations in the csv file description and the reliance on the use of declname as opposed to a proper id for dynamically dispatched methods. One known failure mode of the refactoring tool is when a function (or trait) declared within a scope. Prevented renamings include those on dynamically dispatched methods, handled by the compiler runs on each usage.]

\subsection{Concrete types - structs and enums}
[At the moment there are tests for both renaming of structs and enums with detection of namespace collisions (which are not in local scopes as mentioned above). The checking of namespace collisions also extend to the usage of `use' statements which allow a specific namespace to be added to the default and no need to additionally qualify some names. The renaming of concrete types does not extend to type aliases (or rather the extent to which the renamings function is unknown).]

\subsection{Shared limitations}
All of the current refactorings rely on the fact that the save-analysis output is correct. During the process of this project, this has certainly been found to be false, although not to an extreme extent. Minor errors in processing have meant that little known edge cases have occurred and would not have been found without the testing that has currently been undergone. In terms of testing of the save-analysis functionality, it is relatively scarce and would be difficult to implement considering the different possible combinations of expressions and items. Beyond errors in the save-analysis translation, errors in the compiler would also affect the ability to function correctly, but the compiler in general appears to feature more testing and many of the aspects of the compiler save-analysis in particular depends on are quite critical. On the other hand, lesser used and documented API like name resolution could be affected and this is a problem especially from the context of a third party tool (and the first of its kind to use some API) [which means adoption of this code is all the more important by those who use Rust]. 

\section{Examining the steps necessary to perform a refactoring}

Looking at what is necessary to invoke the refactoring tool we can determine a general description of steps required:

\begin{enumerate}
\item Compile a program with -Z save-analysis to produce a csv analysis file.
\item Either inspect the csv manually or run the refactoring library to determine the node id of the element you wish you alter.
\item Run the refactoring tool, choosing your desired refactoring. With a rename, the new name must be supplied, the node id and (when multi-file refactorings are implemented) the file.
\item Wait for the refactoring to occur and the compiler to run all the proper checks. 
\item Process the result and save the result to disk, if desired.
\end{enumerate}

Ideally, the flow of behaviour a user would want would be to simply identify the row and column and the refactoring to happen automatically. Not involving node id at all would be good, but this is necessary to adequately treat the tool as a library as opposed to a full fledged tool. Being able to easily identify row and column still requires some form of GUI tool and integration with such a tool would be desirable to reduce the amount of different parts required to perform a refactoring. 

Having to regnerate a csv file every time a refactoring has been made is definitely not desirable, although using the same csv file should have no consequences as long as only renamings were performed and renaming occurred on separate variables. By doing so, the compiler runs generate the same AST with the same node ids (is deterministic) and the only difference in the csv output is the naming. Implementing some form of analysis cache would reduce user friction and reduce the amount of time spent compiling. This could be done outside of the refactoring library, in a persisting background process but on the other hand, as soon as a refactoring which significantly changed the output began, any existing analysis would still be out of date. A possible solution would be to correctly implement incremental compilation within the Rust compiler, which would allow less significant compile times and faster regeneration of analysis. The lack of such behaviour is a shortcoming of the compiler and does appear to require significant structual changes but the benefits would extend further than just enabling better refactoring.

\section{Analyzing time taken}
With only single file tests, the amount of time spent performing each refactoring is quite negligible. Compilation times for single files, especially of fairly trivial programs do not provide sufficient evidence of practical timings for performing a refactoring. This is definitely something which should be attempted before the project is completed. 

Typically, the main consumption of compile time is with the analysis phase and generation of LLVM code. Within the refactoring tool, in all cases, the generation of LLVM code is avoided since only source code modifications are being performed and the analysis phase provides more than enough information for validity of a refactoring. While a rename refactoring requires checking every usage of a declared item using the compiler, in the 'happy path' every usage should fail the compiler check during the early stages of parsing or name resolution. Only in more unlikely or unfortunate cases will a refactoring require any additional processing in analysis. In this regard, the compiler approach likely is not as bad as first theorized. On the other hand, the number of runs is still proportional to the number of usages but this is a direct consequence of the current limitations of the name resolution API.

\section{Time taken for a refactoring within a moderate crate?}
\chapter{Future work}\label{C:future}
The work done so far serves as definite proof of the concept and the techniques or approaches used to build the refactoring tool. To ensure that the tool serves practical use however, there is still significant work to be done. The scope of this additional work extends far beyond the time allocated and should serve as an adequate starting point for those looking to extend this work in the future. At this point the goal of this work is to address immediate shortcomings in the tool and to leave the more complicated refactoring and infrastructure changes for others (while providing useful commentary and insight). Lastly, with Rust being an open source project, there is the issue of upstreaming code and being mindful of the fact that the tool should follow as many coding conventions as possible.

\section{Multi-file projects}
At the moment, the refactoring tool only supports single file Rust programs. This is severe limitation in the ability of the tool to function practically since many Rust modules or crates are divided up into multiple files and organized in terms of directories for better code organization. 

Examining the current obstacles for providing this functionality, a separate approach must be taken from the current method using single files. This is brought about by multiple contributing factors. Firstly, rustc typically takes a single argument: the crate root, allowing the remaining files in the crate to be inferred. Due to the current structure of the compiler, these inferred files cannot be intercepted during the compilation process to facilitate the necessary compilation checks for the current renamings. 

The trivial solution to this problem is to modify the files on disk and to generate backup files before a renaming occurs. However, from a personal standpoint this is not a particularly satisfactory solution, especially with the current testing setup. The combination of modifying test files and having common test failures might be extremely cumbersome to deal with, but it may have to be tolerated. Alternatively, a temporary directory may be used to perform these compilation checks, however, ensuring the necessary dependencies are included could prove problematic. Ideally, modifications should be made to the compiler to allow a callback of some description during the parsing and interpretation of crates by the Rust compiler. Due to recent changes in the compiler, this approach appears quite feasible and should be achievable in the short term (1 to 2 weeks).

% Such a change does appear to be quite involved and changing a critical stage in the compiler is likely to present further issues in review and upstream of the code.

Another area of particular interest would be to investigate the overall effect of refactorings across multiple crates and how warnings should be presented when changes occur that might affect a public API. An investigation of Crates.io \cite{cratesio15}, the Rust package repository would likely yield interesting findings and in particular, identification of a description of the steps necessary to take when an API must change and how that should be dealt with from a community perspective. Looking at how the UNIX/Linux community deals with API or ABI changes in their open source software should also prove useful.

\section{Testing}
Going forward, the refactoring tool needs significantly more testing to ensure that the current set of refactorings are correct. In this scenario testing has an even more important role, it is to ensure that the existing refactorings are not broken with further changes to the tool. With the existing drastic changes to the approach required to progress to this stage, it appears likely that such changes are to be expected. Furthermore, the refactoring tool evolves independently of the Rust compiler and changes to the compiler are likely to affect the inevitable outcome of the refactoring tool. Already, a number of existing bugs have been found and reintroducing them by accident does not appear to be difficult with current limitations in the amount of testing and documentation available for some areas of the compiler. With unstable API it is unsurprising and so in this manner, expanding testing does not necessarily need to be confined to the refactoring tool. Tests should continue to be added for the remainder of the project to eliminate as many edge cases as possible.

With limited experience with Rust in general, using real world code to attempt refactorings would help test for more obscure corner cases which may have been overlooked. Crates.io \cite{cratesio15} should prove an invaluable resource for acquiring Rust code to perform such testing. Furthermore, initial analysis of the effiency of the tool can be produced and speculations as to the usefulness. The current testing programs are small and limited and by using real world code and by taking timing measurements, assessments can be made about how much faster the tool would be ideally. This should be possible after multi-file support is completed and should not pose any significant further issues, taking place during week 3 or 4 of Trimester 2.

\section{Future refactorings}
In terms of future refactorings, there are a number of next-step refactorings which should use most of the same infrastructure of the existing tool. For instance, struct fields or function arguments should function similarly to local variables and trait renaming should use most of the same conventions provided by concrete types like structs or enums. This is because of the output of save-analysis and the simplifications made in process of generating the csv output. A handful of these should be implemented relatively quickly, under a days work, however ensuring testing is adequate should take longer. As a whole, as many as possible should be implemented to provide a more comprehensive tool, however due to their simplicity they should not be the highest priority.

Renaming and modifying type aliases should also present interesting problems due to deviations from typical object-orientated languages, along with type parameters and associated types. Furthermore, as previously described, renaming types declared within some lexical scope or block introduce issues to do with path resolution that still need to be resolved. To do so requires determining which parts of the path are `private' to that scope and should not be used in the general case. The treatment of types in Rust definitely complicates matters, and this is already with the simplifications made from save-analysis.

In the context of Rust, there are a number of refactorings which would be more unique. Extraction of a method, as described by Fowler, presents a number of intricacies tied with lifetimes and the ownership system in Rust. Ensuring borrows are made correctly as well as dealing with move semantics if a constructor is within the extracted section of code present difficulties relatively specific to Rust refactoring. While other refactoring changes are semantics preserving, extraction of a method could introduce new lifetimes and verification of correctness may not be so trivial. Similarly, extraction and inlining of local variables also appears to provide adequate difficulties with the ownership system and should be able to provide further insight into approaching method extraction. Researching as much as possible the potential for code extraction should provide more depth to the analysis made by this project and should comprise the majority of the remaining time spent. Achieving some form of code extraction would be ideal, but regardless the findings made should also provide noteworthy material.

Converting functions to methods or vice versa is also a critical function that would be useful for the Rust community. Such a change is common with unstable API and within Rust, although released, there are a number of unstable API. To alleviate issues with changes, supporting these changes with a refactoring tool would be incredibly useful, even if it only used a structural search and replace. This would provide functionality similar to gofmt and go-fix as Go originally updated their API \cite{gofix11}. Making changes which break a sizable amount of user-code is unfavourable, however, it usually must be done at some point and haivng a tool to remedy the stress of such a change would be good for both the developers of Rust and the community.

\noindent
A non-comprehensive list of further refactorings that might be considered are:
\begin{itemize}
\item Inlining methods
\item Inlining modules
\item Creating modules
\item Adding or removing arguments to functions
\item Moving inherited to trait
\item Extracting trait
\item Changing types to type aliases
\end{itemize}

% List a number of refactorings which would be interesting in the context of Rust
% Comparing closures to other languages
% Extract a global
% Constructor to factory?
% Unknown the extent to which refactoring works in unsafe code blocks

\section{Additional compiler improvements}
% -Z save-analysis
Given the current architecture of the tool, there are a number of further improvements that could be made to the compiler to improve its efficacy and efficiency. Using -Z save-analysis provides another additional step before a refactoring can be made. Furthermore, it must be generated every time a refactoring is to be undergone. By providing a library and API with which this information can be queried using the existing compiler API, the need for CSV parsing and associated unrelated complexity disappears. The actual performance offset of having to run the compiler again to generate this information does not disappear however.  By running all the way through to analysis, this still takes a significant amount of time. 

%[reference to compiler stages from earlier section]

%  Compiler speed-up + Name resolution at the lowest level
Helping to address performance in the compiler itself in the general case should also help to improve performance of the refactoring tool. However, based on the current design of the tool and the need for multiple runs with modified source, the issue of performance would be better addressed by improvements to the name resolution module in the Rust compiler. In particular, providing name resolution for usages instead of just at the declaration level would turn O(N) runs of the compiler into O(1) runs by using name resolution multiple times in a single run. Fundamentally this requires heavy changes to the structure of the compiler and the nature of name resolution. As it currently is, not every node gets its individual node id in nested expressions and to resolve this requires significantly more memory usage or novel, significant and well-architected changes, well beyond the scope of this project.

% Macro hygiene with types
% Renaming within macros - seems like a big problem

Currently macros are completely ignored by the save-analysis report generated by the compiler. As it is, this appears to be a major shortcoming in the refactoring tool that has yet to be addressed. However, not a lot can be done at this point due to the general limitations regarding macros and how they are typically ignored when generating metadata from the compiler. Without being able to identify any spans associated with macros, it isn't possible to make the necessary code transformations. Implementing the necessary functionality is certainly non-trivial and likely requires much better domain knowledge than that which could be provided here.

\section{Miscellaneous}
% Refactoring practical...
A user-study or fielding comments on the tool would provide some interesting feedback. In order to provide usefulness, this is incredibly important part of assessing a tool. Furthermore production of a prototype GUI tool or Emacs plugin could increase the amount of users of the tool, or people interested in the tool. That could generate interest in further development and further research began by this project.

% Macro hygiene vs refactoring
The relationship between macro hygiene and refactoring is particularly novel, but from initial analysis, does not appear to provide any particular benefits. Further research into the relationship might provide some unique insights and a system which is able to incorporate both would be of significant academic interest, and interest to this author.

% Clang map reduce - Google scale of refactoring necessary. Analysis of speed in general.
Effiency of large-scale refactoring in general needs more attention. Although analysis of individual packages on a service like Crates.io provides some benefits, it would be curious to see how a tool functions on code bases which are much larger, in the order of hundreds of thousands of lines of code or larger. Although the expectations on this particular tool are not quite so high, understanding the general tradeoffs of provably correct refactorings and time taken and developer perception of this tradeoff is likely to produce fascinating observations. Looking at the Google-scale of refactoring, Google have set up a Clang map-reduce to perform refactoring on large codebases over a network of machines \cite{carruth2011clang}. Evidently the utility and associated confidence in such refactorings led them to spend the time to create such a framework, but determining when this happens in the general case provides provocative food-for-thought.


%\include{example}
\chapter{Conclusion}\label{C:con}
% Documentation of API is necessary.
This work has explored refactoring and providing tool-support within the context of the Rust programming language. Utilizing existing infrastructure provided by the compiler, this work identifies extensions which help facilitate common automated refactorings. In a more general sense, this work attempts to build upon existing work done on refactoring by documenting specific decisions made in building a refactoring tool (which might only encoded in the source code of an actual tool), and attempting to analyze the decisions made by others.

% Rust provides some unique obstacles in providing automated refactorings...

At the moment the provided tool supports renaming of local and global variables, fields, function arguments, structs, enum and functions. Beyond renaming, it also allows reification and elision of lifetime parameters for functions and methods and has preliminary support for inlining of local variables. However, the complete limitations of these refactorings are not yet fully known, but there exists a current suite of tests to ensure that there are no obvious flaws in the approach. The presence of bugs in the compiler is a real problem to generality, but without real world use and more contribution to testing, finding these corner cases appears to be difficult.

At the moment, Rust lacks significant refactoring tool-support and evidently requires more work particularly within the compiler to enable further, valuable progress. Although a preliminary tool has been provided, there are many avenues for continuing work and the hope is that this first investigation provides useful insight for future efforts. Understanding the required context and the necessary infrastructure has been a major part of this work, and continuation should allow greater focus on implementing a more difficult and comprehensive set of refactorings for Rust.

\chapter{Appendix}\label{C:appen}
\section{Performance data}

\subsection{Relative crate sizes}
Figure \ref{Fig:codesize} lists the four crates from Crates.io that have been chosen to help evaluate the produced tool. The listed crates form four of the top five most downloaded crates by the Rust community \cite{cratesio15}. The `winapi' crate was omitted due to less relevance on a Linux platform. A fifth crate `bitflags' was originally going to be used for analysis; however macro incompatibility made this an impossible task. The lines of code metric only concerns Rust source (.rs) files and does not take into account comments or test code. The purpose of the comparison is only to generate a rough, high level contrast and to gather any overall insights. 

\begin{figure}[H]
\begin{center}
    \begin{tabular}{ | l | c |}
    \hline
    \textbf{Rust crate} & \textbf{Lines of code} \\ \hline
    libc & 6547 \\ \hline
    rustc-serialize &  5741 \\ \hline
    rand &   5187 \\ \hline
    log &  1449 \\ \hline
    \end{tabular}
\end{center}

\caption{General figure for the relative size of crates compared}
\label{Fig:codesize}
\end{figure}

\subsection{Timings for the different refactorings}
Timings were generated for the different crates using the Linux perf tool: {\verb|perf stat -r 10|} Timings were averaged over 10 runs and use of the perf tool gave much less unaccountable variations in results compared to other tools such as {\verb|time|}. The machine used was a dual core 2.0 GHz virtual machine running Ubuntu 12.04 with Rust Nightly 23.09.2015 (along with the latest version of the refactoring tool). The tool was compiled in release mode (not debug) which ensures increased speed, by at least 10 times based on observation. Timings represent elapsed time, not system time or CPU time.

The classes of refactorings measured are: renaming variables (or variable-like constructs), renaming functions (or methods), renaming concrete types, reification and elision. Inline local has been omitted due to lack of sufficient examples in the given crates. In each case, examples were picked with effectively a single usage (minimal modification) so that the difference in timings between each of the individual refactorings could be highlighted. 

\begin{figure}[H]
\begin{center}
    \begin{tabular}{ | l | c |}
    \hline
    \textbf{Refactoring} & \textbf{Time in seconds} \\ \hline
    Single variable usage & 0.279296164 \\ \hline
    Single function usage &  0.282359402  \\ \hline
    Single concrete type usage  &  0.192517137 \\ \hline
    Reify function &  0.181052082 \\ \hline
    Elide function &  0.256281695 \\ \hline
    \end{tabular}
\end{center}

\caption{Timings for libc}
\label{Fig:libc}
\end{figure}

\begin{figure}[H]
\begin{center}
    \begin{tabular}{ | l | c |}
    \hline
    \textbf{Refactoring} & \textbf{Time in seconds} \\ \hline
    Single variable usage &  1.190142460 \\ \hline
    Single function usage &  1.765529977  \\ \hline
    Single concrete type usage  &  1.291176595 \\ \hline
    Reify function &  0.961171871  \\ \hline
    Elide function &  0.948008818 \\ \hline
    \end{tabular}
\end{center}

\caption{Timings for rustc-serialize}
\label{Fig:rustc-serialize}
\end{figure}

\begin{figure}[H]
\begin{center}
    \begin{tabular}{ | l | c |}
    \hline
    \textbf{Refactoring} & \textbf{Time in seconds} \\ \hline
    Single variable usage &  0.527383479 \\ \hline
    Single function usage &  0.755735400 \\ \hline
    Single concrete type usage  & 0.555294948 \\ \hline
    Reify function &   0.358056910 \\ \hline
    Elide function & 0.460453089 \\ \hline
    \end{tabular}
\end{center}

\caption{Timings for rand}
\label{Fig:rand}
\end{figure}

\begin{figure}[H]
\begin{center}
    \begin{tabular}{ | l | c |}
    \hline
    \textbf{Refactoring} & \textbf{Time in seconds} \\ \hline
    Single variable usage &  0.381467818  \\ \hline
    Single function usage &   0.402882117  \\ \hline
    Single concrete type usage  &  0.363789533 \\ \hline
    Reify function &   0.317281525 \\ \hline
    Elide function &  0.324042647 \\ \hline
    \end{tabular}
\end{center}

\caption{Timings for log}
\label{Fig:log}
\end{figure}

\begin{figure}[H]
\begin{center}
\scalebox{0.8}{
\includegraphics{refactorings}
}
\caption{Graph displaying results of the different refactorings}
\label{Fig:compareref}
\end{center}
\end{figure}

\subsection{Scalability of renames}
Since the renamings should follow the same general pattern inside the tool, exploring variation with a single type of renaming on a single crate should show the general relationship. In particular, `libc' was chosen for having a wider variety of occurrence counts specifically with concrete types. Other crates or types of renamings had few examples and only a limited amount of variation.

\begin{figure}[H]
\begin{center}
    \begin{tabular}{ | l | c |}
    \hline
    \textbf{Number of replaced occurrences} & \textbf{Time in seconds} \\ \hline
    1 type usage &  0.192517137  \\ \hline
    3 type usages &  0.382087255  \\ \hline
    4 type usages &   0.474888942  \\ \hline
    6 type usages &   0.673731962  \\ \hline
    8 type usages &   0.873888499 \\ \hline
    13 type usages  &  1.372581328 \\ \hline
    29 type usages &  2.960479041  \\ \hline
    51 type usages &  5.059021681 \\ \hline
    \end{tabular}
\end{center}

\caption{Timings for varying usage counts of concrete types in libc}
\label{Fig:scaling}
\end{figure}

\begin{figure}[H]
\begin{center}
\scalebox{0.8}{
\includegraphics{scaling}
}
\caption{Graph displaying results of varying the number of usages}
\label{Fig:comparerefs}
\end{center}
\end{figure}


%%%%%%%%%%%%%%%%%%%%%%%%%%%%%%%%%%%%%%%%%%%%%%%%%%%%%%%

\backmatter

%%%%%%%%%%%%%%%%%%%%%%%%%%%%%%%%%%%%%%%%%%%%%%%%%%%%%%%


%\bibliographystyle{ieeetr}
\bibliographystyle{acm}
\bibliography{sample}
%\bibliography{real}


\end{document}
